%https://gist.github.com/CapacitorSet/a5c1b0e50560e89ce994

\documentclass{article}
\usepackage[utf8]{inputenc}
\usepackage{amsmath}
\title{Study of a population system}
\begin{document}

\maketitle

\section{Abstract}
It is our goal to create and run a digital simulation of a population with known, configurable feature, which will later be formally studied on a per-individual and global basis.

\section{Implementation}
This section discusses a Java implementation of the system.\\

We have decided to represent the system not as a whole, but rather as a list of "items", each being updated upon a "game tick".

\subsection{Movement}
The direction is determined by a gravity-like algorithm, with some constant $K$ determining the speed. Specifically, the acceleration $\mathbf a$ in a given axis (eg. $x$), towards an individual $i$, is:
\begin{equation}
\label{eqn:acc_singola}
\mathbf{a}_x(i) = \frac{\Delta x}{\text{Distance}}*G*\frac{i_\text{weight}}{\text{Distance}^2}
\end{equation}
The reader may recognize the above formula as Newton's law of gravity, where $m_1=1$ and $m_2$ is replaced by $i_\text{weight}$. $i_\text{weight}$ is a factor calculated upon each tick. For example, if an individual is thirsty, water sources will have a high $i_\text{weight}$.\\
The total acceleration $\mathbf A$ in a given axis (eg. $x$) is:
\begin{equation}
\label{eqn:acc_totale}
\mathbf{A}_x = \sum_i \mathbf{a}_x(i)
\end{equation}
i.e. the sum of all accelerations.

\subsection{Thirst}
Each individual has a "thirst" variable, which determines its attraction towards water sources. Upon reaching a water source, its thirst becomes negative; thirst increases linearly with time, following this formula:
\[
\frac{d}{dt}\text{Thirst} = K
\]
where $K$ is some constant.\\
The weight $w$ of water sources used in the computation of movement is related to thirst:
\[
w = \begin{cases}
0&\text{for Thirst}<0\\
\text{Thirst}&\text{else}
\end{cases}
\]
The above formula can be understood as: "If the individual is thirsty, it seeks (is attracted to) water sources; if the individual is not thirsty ($\text{Thirst}<0$), it neither seeks, \textit{nor avoids} water sources."\\
If the function was simply $w = \text{Thirst}$, individuals would actively avoid water sources when not thirsty, and this is not desirable.
\section{Formal study}
This section provides a simple analysis of the system, based on the study of the algorithms employed.

\subsection{Movement}
\paragraph{Variables} $x$, $y$, $V_x$, $V_y$\\
\\
\[
\frac{dx}{dt}=V_x
\]
\[
\frac{dy}{dt}=V_y
\]
The algorithms presented in the "Implementation" (\ref{eqn:acc_singola}, \ref{eqn:acc_totale}) were defined in terms of acceleration; to create the mathematical model, it is sufficient to rewrite them in terms of "derivative of velocity".
\[
\frac{\partial}{\partial t}V_x(i) = \mathbf{a}_x(i) = \frac{\Delta x}{\text{Distance}}*G*\frac{i_\text{weight}}{\text{Distance}^2}
\]

\[
\frac{\partial}{\partial t}V_x = \mathbf{A}_x = \sum_i \frac{\partial}{\partial t}V_x(i)
\]

\subsection{Thirst}
\paragraph{Variables} Thirst, weight $w$ of water sources\\
\\
\[
\frac{d}{dt}\text{Thirst} = K
\]
\[
w = \begin{cases}
0&\text{for Thirst}<0\\
\text{Thirst}&\text{else}
\end{cases}
\]
\end{document}
